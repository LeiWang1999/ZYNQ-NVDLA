\chapter{软件设计实现}\label{chap:software}

\section{NVDLA 软件工具链概述}

在前面的章节,我们略微提到过 NVDLA 的软件栈。但是在这一小节,我们将详细的介绍 NVDLA 的软件工具链。如图~\ref{fig:NVDLA Software},英伟达官方提供了完整的软件生态。

\begin{figure}[!htbp]
    \centering
    \includegraphics[width=0.7\textwidth]{software_package.png}
    \caption{NVDLA Software}
    \label{fig:NVDLA Software}
\end{figure}

Compiler 是软件工具链的前端,与硬件无关,Compiler 能够接受的参数如下:

\begin{lstlisting}
./nvdla_compiler -h
Usage: ./nvdla_compiler [-options] --prototxt <prototxt_file> --caffemodel <caffemodel_file>
where options include:
    -h                                                 print this help message
    -o <outputpath>                                    
    --profile <basic|default|performance|fast-math>    computation profile
    --cprecision <fp16|int8>                           compute precision
    --configtarget <nv_full|nv_large|nv_small>         target platform
    --calibtable <int8 calib file>                     
    --quantizationMode <per-kernel|per-filter>         
\end{lstlisting}

\begin{itemize}
    \item profile 指定了优化方案,在 Compiler 内部提供了四种优化方案,能够支持一些硬件无关的网络优化,例如算子融合、内存重用等,本设计使用默认的 fast-math,启用全部的优化选项。
    \item cprecision 指定了精度,在这里我们需要选择 int8,并且可以结合 TensorRT 生成 calibtabel 文件,如果不给出 calibtabel 参数,Compiler 内部会使用简单的量化方案,实测效果很差。
    \item configtarget 本设计选择 nv\_small,匹配本设计的 small 配置。
    \item calibtabel 需由 TensorRT 生成,将在下一小节详细介绍。
    \item quantizationMode 有两个选项,per-kernel 选项是对每一层卷积使用相同的量化参数,per-filter 则是对每一层卷积使用不同的量化参数,这需要结合 calibtabel 文件选择,在本设计中使用 per-kernel。
\end{itemize}

最终,Compiler 生成 Loadable 文件,交给 Runtime 进行加速器的调度。

Loadable 文件是 Compiler 与 Runtime 之间通信的媒介,其由 Google 开源的 FlatBuffers 序列化协议所组织,能够将对象与数据进行压缩,以便在网络中进行传输。简单来讲,我们需要在流中传输一个对象,比如网络流。一般我们需要把这个对象序列化之后才能在流中传输(例如,可以把对象直接转化为字符串),然后在接收端进行反序列化(例如把字符串解析成对象)。但是显然把对象转成字符串传输的方法效率十分低下,于是有了各种流的转换协议,FlatBuffers也是其中一种。

Runtime 与硬件紧密贴合,其又分为 UMD 和 KMD 两个部分:UMD 是用户应用,需要我们在 Linux 上编译运行,其接受 Loadable 文件并解析,最后递交一个推理任务到 KMD;KMD 是内核驱动,需要我们在构建 Linux 的时候编译,运行 Linux 的时候挂载,接受推理任务之后负责调度网络,配置寄存器,处理中断等任务。Runtime 能够接受的参数如下:

\begin{lstlisting}
./nvdla_runtime -h
Usage: ./nvdla_runtime [-options] --loadable <loadable_file>
where options include:
    -h                    print this help message
    -s                    launch test in server mode
    --image <file>        input jpg/pgm file
    --normalize <value>   normalize value for input image
    --mean <value>        comma separated mean value for input image
    --rawdump             dump raw dimg data
\end{lstlisting}

Compiler 的编译过程较简单,本设计不多介绍,Runtime 的构建过程非常复杂,将在后面的章节详细介绍。

\section{TensorRT 与模型量化}

本设计使用的 small 配置,仅支持 INT8 的推理,而 Caffe 框架仅支持 Float32 类型的训练,所以我们必须进行模型的量化。前文提到,在 Compiler 中量化需要结合英伟达公司的 TensorRT 框架。

\subsection{TensorRT 量化原理}

将高精度的浮点型转化为八比特的定点类型的量化方法,都遵循如下的公式:

$$ FP32 Tensor (T) = scale\_factor(sf) * 8-bit Tensor(t) + FP32\_bias (b) $$

通过实验测得,bias去掉对精度的影响不是很大,最终变为:

$$ T = sf * t $$

\subsection{量化步骤与精度损失}

\subsection{calibtabel 生成}

\section{Petalinux 工具介绍}

\section{Ubuntu 16.04 根文件系统移植}

\subsection{读取 Block Design 配置信息}

\subsection{Linux 内核裁剪}

\subsection{新增 Linux 设备树节点}

\subsection{生成 Boot 和 Image 文件}

\subsection{替换根文件系统}

\section{KMD 内核程序挂载}

\section{UMD 应用程序编译}
