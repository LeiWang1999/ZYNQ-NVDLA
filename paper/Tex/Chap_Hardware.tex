\chapter{硬件设计实现}\label{chap:hardware}


\section{Xilinx FPGA 设计套件}

在上一章节中提到,本设计选用的芯片信号为 XC7Z045-2FFG900I,其为著名的 FPGA 供应厂商 Xilinx 研发。Xilinx 提供了非常完整的可编程逻辑开发软件工具链,及 Vivado 设计套件,使用该套件可以将我们的 Verilog 文件封装成知识产权 IP,进行硬件电路的仿真,生成比特流与硬件描述文件等功能。使用 ZYNQ 器件,也需要我们在 Vivado 的 BlockDesign 页面中分配 ARM 处理器使用到的资源。除此之外,Vivado 设计套件还包括了开发 ARM 裸机的 Vivado SDK、能够基于 LLVM 将 C$\backslash$C++ 程序转换为底层逻辑电路的 Vivado HLS 等应用。

本章节将说明如何使用 Vivado 设计套件完成 NVDLA IP 的打包,Block Design 设计,查看资源利用、时序、功耗等报告,最后生成比特流文件与硬件描述文件。 

\section{NVDLA IP 生成描述}

这一小节将详细介绍如何修改 NVDLA 的配置文件,以及根据配置文件生成 NVDLA 的 RTL 代码,并将 NVDLA 的接口进行封装,打包成 IP。

总的来说,生成 NVDLA 的 RTL 代码有两个路径,其一是英伟达官方提供的 hw 项目,可以根据预先定义好的 spec 文件生成,但是这个工作流程需要预先构建好官方提供的 \emph{make} 工具,该工具又依赖 GCC、Java、Perl、Verilator、Python 等,稍显复杂。第二条路径是使用一名由伯克利大学的研究生使用 Chisel 语言编写的 NVDLA 项目,其也是可以生成 RTL 代码的,但是由 Chisel 生成的 RTL 代码会被压缩在一个文件之内,NVDLA 的代码输出高达数万行,不利于阅读与分析。

综合以上,本文使用 \emph{make} 进行 RTL 生成,并且使用 Docker 容器技术分离环境解决 \emph{make} 软件依赖过复杂的问题。

\subsection{基于 \emph{make} 的 RTL 生成}

为了能够在不污染主机环境的情况下构建 \emph{make} 工具,本设计使用 Docker 容器技术搭建软件环境,基于 Ubuntu:16.04 容器,分别安装好如下环境:

\begin{itemize}
    \item GCC 4.8.5
    \item OpenJDK 1.8.0
    \item SystemC 2.3.0
    \item Python 2.7.12
    \item VCS 2016.06
    \item Verilator 3.912
    \item Clang 3.8.0
\end{itemize}

安装好环境之后,我们可以通过新建一个 spec 文件来修改NVDLA的配置。例如通过定义宏 FEATURE\_DATA\_TYPE\_INT8 可以指定 Feature Map 的数据格式为 INT8;通过定义宏 WINOGRAD\_DISABLE 来关闭 WINOGRAD 卷积电路。本设计采用的是官方提供的最小配置,及使用 small.spec 文件。

有关 small.spec 的详细内容详见附录,在使用 \emph{tmake} 之前,我们还需要在根目录通过 \emph{make} 命令选择 small.spec 为配置文件,并输入安装的软件生态的可执行文件路径,该 \emph{make} 命令最终会生成一个 tree.make 的文件,\emph{tmake} 工具的本质是根据与现实先定义好的宏,将模板文件夹 vmod 里的 RTL 代码进行文本预处理,将不需要的文本删除输出。

\lstset{language=Bash}
\begin{lstlisting}
./tools/bin/tmake -build vmod
\end{lstlisting}

这样就可以生成可以被综合的RTL代码了。

\subsection{csb2apb}

\subsection{关闭 Clock Gating}

\subsection{增加 Address Block}

\subsection{IP Package 封装AXI总线}

\section{Top Block Design 设计}

\subsection{APB to AXI Bridge}

\subsection{AXI Smart Connect}

\subsection{ZYNQ 7000+ IP}

\section{综合}

\subsection{资源利用率报告}

\subsection{Timing 报告}

\subsection{功耗报告}

\section{实现}

\section{Bitstream 与 hdf 文件生成}










