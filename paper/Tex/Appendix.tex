\chapter{}

\section{\emph{small.spec}}

\lstset{language=C++}
\begin{lstlisting}
#define FEATURE_DATA_TYPE_INT8
#define WEIGHT_DATA_TYPE_INT8
#define WEIGHT_COMPRESSION_DISABLE
#define WINOGRAD_DISABLE 
#define BATCH_DISABLE
#define SECONDARY_MEMIF_DISABLE
#define SDP_LUT_DISABLE
#define SDP_BS_ENABLE
#define SDP_BN_ENABLE
#define SDP_EW_DISABLE
#define BDMA_DISABLE
#define RUBIK_DISABLE
#define RUBIK_CONTRACT_DISABLE
#define RUBIK_RESHAPE_DISABLE
#define PDP_ENABLE
#define CDP_ENABLE
#define RETIMING_DISABLE
#define MAC_ATOMIC_C_SIZE_8
#define MAC_ATOMIC_K_SIZE_8
#define MEMORY_ATOMIC_SIZE_8
#define MAX_BATCH_SIZE_x
#define CBUF_BANK_NUMBER_32
#define CBUF_BANK_WIDTH_8
#define CBUF_BANK_DEPTH_512
#define SDP_BS_THROUGHPUT_1
#define SDP_BN_THROUGHPUT_1
#define SDP_EW_THROUGHPUT_x
#define PDP_THROUGHPUT_1
#define CDP_THROUGHPUT_1
#define PRIMARY_MEMIF_LATENCY_64
#define SECONDARY_MEMIF_LATENCY_x
#define PRIMARY_MEMIF_MAX_BURST_LENGTH_1
#define PRIMARY_MEMIF_WIDTH_64
#define SECONDARY_MEMIF_MAX_BURST_LENGTH_x
#define SECONDARY_MEMIF_WIDTH_x
#define MEM_ADDRESS_WIDTH_32
#define NUM_DMA_READ_CLIENTS_7
#define NUM_DMA_WRITE_CLIENTS_3

#include "projects.spec"

\end{lstlisting}


\section{\emph{Nv\_nvdla\_wrapper.v}}

\lstset{language=Verilog}
\begin{lstlisting}
module NV_nvdla_wrapper(
    input core_clk,
    input csb_clk,
    input rstn,
    input csb_rstn,

    output dla_intr,
    // dbb AXI
    output nvdla_core2dbb_aw_awvalid,
    input nvdla_core2dbb_aw_awready,
    output [7:0] nvdla_core2dbb_aw_awid,
    output [3:0] nvdla_core2dbb_aw_awlen,
    output [2:0] nvdla_core2dbb_aw_awsize,
    output [64 -1:0] nvdla_core2dbb_aw_awaddr,
    output nvdla_core2dbb_w_wvalid,
    input nvdla_core2dbb_w_wready,
    output [64 -1:0] nvdla_core2dbb_w_wdata,
    output [64/8-1:0] nvdla_core2dbb_w_wstrb,
    output nvdla_core2dbb_w_wlast,
    output nvdla_core2dbb_ar_arvalid,
    input nvdla_core2dbb_ar_arready,
    output [7:0] nvdla_core2dbb_ar_arid,
    output [3:0] nvdla_core2dbb_ar_arlen,
    output [2:0] nvdla_core2dbb_ar_arsize,
    output [64 -1:0] nvdla_core2dbb_ar_araddr,
    input nvdla_core2dbb_b_bvalid,
    output nvdla_core2dbb_b_bready,
    input [7:0] nvdla_core2dbb_b_bid,
    input nvdla_core2dbb_r_rvalid,
    output nvdla_core2dbb_r_rready,
    input [7:0] nvdla_core2dbb_r_rid,
    input nvdla_core2dbb_r_rlast,
    input [64 -1:0] nvdla_core2dbb_r_rdata,
    output [1:0] m_axi_awburst,
    output  m_axi_awlock, 
    output [3:0] m_axi_awcache,
    output [2:0] m_axi_awprot, 
    output [3:0] m_axi_awqos,  
    output  m_axi_awuser, 
    output  m_axi_wuser,  
    input  [1:0] m_axi_bresp,
    input   m_axi_buser,
    output [1:0] m_axi_arburst,
    output  m_axi_arlock, 
    output [3:0] m_axi_arcache,
    output [2:0] m_axi_arprot, 
    output [3:0] m_axi_arqos,  
    output  m_axi_aruser, 
    input  [1:0] m_axi_rresp,
    input   m_axi_ruser,
    // cfg APB
    input psel,
    input penable,
    input pwrite,
    input [31:0] paddr,
    input [31:0] pwdata,
    output [31:0] prdata,
    output pready,
    output pslverr
    );

    wire        m_csb2nvdla_valid;
    wire        m_csb2nvdla_ready;
    wire [15:0] m_csb2nvdla_addr;
    wire [31:0] m_csb2nvdla_wdat;
    wire        m_csb2nvdla_write;
    wire        m_csb2nvdla_nposted;
    wire        m_nvdla2csb_valid;
    wire [31:0] m_nvdla2csb_data;


    NV_NVDLA_apb2csb apb2csb (
        .pclk                  (csb_clk)
        ,.prstn                 (csb_rstn)
        ,.csb2nvdla_ready       (m_csb2nvdla_ready)
        ,.nvdla2csb_data        (m_nvdla2csb_data)
        ,.nvdla2csb_valid       (m_nvdla2csb_valid)
        ,.paddr                 (paddr)
        ,.penable               (penable)
        ,.psel                  (psel)
        ,.pwdata                (pwdata)
        ,.pwrite                (pwrite)
        ,.csb2nvdla_addr        (m_csb2nvdla_addr)
        ,.csb2nvdla_nposted     (m_csb2nvdla_nposted)
        ,.csb2nvdla_valid       (m_csb2nvdla_valid)
        ,.csb2nvdla_wdat        (m_csb2nvdla_wdat)
        ,.csb2nvdla_write       (m_csb2nvdla_write)
        ,.prdata                (prdata)
        ,.pready                (pready)
    );


    NV_nvdla nvdla_top (
        .dla_core_clk                    (core_clk)
        ,.dla_csb_clk                     (csb_clk)
        ,.global_clk_ovr_on               (1'b0)
        ,.tmc2slcg_disable_clock_gating   (1'b0)
        ,.dla_reset_rstn                  (rstn)
        ,.direct_reset_                   (1'b1)
        ,.test_mode                       (1'b0)
        ,.csb2nvdla_valid                 (m_csb2nvdla_valid)
        ,.csb2nvdla_ready                 (m_csb2nvdla_ready)
        ,.csb2nvdla_addr                  (m_csb2nvdla_addr)
        ,.csb2nvdla_wdat                  (m_csb2nvdla_wdat)
        ,.csb2nvdla_write                 (m_csb2nvdla_write)
        ,.csb2nvdla_nposted               (m_csb2nvdla_nposted)
        ,.nvdla2csb_valid                 (m_nvdla2csb_valid)
        ,.nvdla2csb_data                  (m_nvdla2csb_data)
        ,.nvdla2csb_wr_complete           () //FIXME: no such port in apb2csb
        ,.nvdla_core2dbb_aw_awvalid       (nvdla_core2dbb_aw_awvalid)
        ,.nvdla_core2dbb_aw_awready       (nvdla_core2dbb_aw_awready)
        ,.nvdla_core2dbb_aw_awaddr        (nvdla_core2dbb_aw_awaddr)
        ,.nvdla_core2dbb_aw_awid          (nvdla_core2dbb_aw_awid)
        ,.nvdla_core2dbb_aw_awlen         (nvdla_core2dbb_aw_awlen)
        ,.nvdla_core2dbb_w_wvalid         (nvdla_core2dbb_w_wvalid)
        ,.nvdla_core2dbb_w_wready         (nvdla_core2dbb_w_wready)
        ,.nvdla_core2dbb_w_wdata          (nvdla_core2dbb_w_wdata)
        ,.nvdla_core2dbb_w_wstrb          (nvdla_core2dbb_w_wstrb)
        ,.nvdla_core2dbb_w_wlast          (nvdla_core2dbb_w_wlast)
        ,.nvdla_core2dbb_b_bvalid         (nvdla_core2dbb_b_bvalid)
        ,.nvdla_core2dbb_b_bready         (nvdla_core2dbb_b_bready)
        ,.nvdla_core2dbb_b_bid            (nvdla_core2dbb_b_bid)
        ,.nvdla_core2dbb_ar_arvalid       (nvdla_core2dbb_ar_arvalid)
        ,.nvdla_core2dbb_ar_arready       (nvdla_core2dbb_ar_arready)
        ,.nvdla_core2dbb_ar_araddr        (nvdla_core2dbb_ar_araddr)
        ,.nvdla_core2dbb_ar_arid          (nvdla_core2dbb_ar_arid)
        ,.nvdla_core2dbb_ar_arlen         (nvdla_core2dbb_ar_arlen)
        ,.nvdla_core2dbb_r_rvalid         (nvdla_core2dbb_r_rvalid)
        ,.nvdla_core2dbb_r_rready         (nvdla_core2dbb_r_rready)
        ,.nvdla_core2dbb_r_rid            (nvdla_core2dbb_r_rid)
        ,.nvdla_core2dbb_r_rlast          (nvdla_core2dbb_r_rlast)
        ,.nvdla_core2dbb_r_rdata          (nvdla_core2dbb_r_rdata)
        ,.dla_intr                        (dla_intr)
        ,.nvdla_pwrbus_ram_c_pd           (32'b0)
        ,.nvdla_pwrbus_ram_ma_pd          (32'b0)
        ,.nvdla_pwrbus_ram_mb_pd          (32'b0)
        ,.nvdla_pwrbus_ram_p_pd           (32'b0)
        ,.nvdla_pwrbus_ram_o_pd           (32'b0)
        ,.nvdla_pwrbus_ram_a_pd           (32'b0)
    ); // nvdla_top

assign nvdla_core2dbb_aw_awsize = 3'b011;
assign nvdla_core2dbb_ar_arsize = 3'b011;

assign m_axi_awburst = 2'b01;
assign m_axi_awlock  = 1'b0;
assign m_axi_awcache = 4'b0010;
assign m_axi_awprot  = 3'h0;
assign m_axi_awqos   = 4'h0;
assign m_axi_awuser  = 'b1;
assign m_axi_wuser   = 'b0;
assign m_axi_arburst = 2'b01;
assign m_axi_arlock  = 1'b0;
assign m_axi_arcache = 4'b0010;
assign m_axi_arprot  = 3'h0;
assign m_axi_arqos   = 4'h0;
assign m_axi_aruser  = 'b1;

assign pslverr = 1'b0;

endmodule
\end{lstlisting}

\section{\emph{Lenet5 量化关键代码}}

\lstset{language=Python}
\begin{lstlisting}
# Returns a numpy buffer of shape (num_images, 1, 28, 28)
def load_data(filepath):
    test_imgs = []
    global fileList
    fileList = os.listdir(filepath)
    mean = np.ones([3, 224, 224], dtype=np.float)
    mean[0,:,:] = 104
    mean[1,:,:] = 117
    mean[2,:,:] = 123
    for img_path in fileList:
        img_path = filepath +'/'+ img_path
        img = cv.imread(img_path)
        img = crop_img(img, [224, 224])
        img = img.transpose((2, 0, 1))
        img = img - mean
        test_imgs.append(img)
    # Need to scale all values to the range of [0, 1]
    return np.ascontiguousarray(test_imgs).astype(np.float32)

# Returns a numpy buffer of shape (num_images)
def load_labels(filepath):
    global fileList
    test_labels = []
    labels_mapping = {}
    with open(filepath, 'r') as f:
        lines = f.readlines()
        for each in lines:
            imageName, labels = each.strip('\n').split(' ')
            labels_mapping[imageName] = int(labels)
    for each in fileList:
        test_labels.append(labels_mapping[each])
    return np.ascontiguousarray(test_labels)
\end{lstlisting}

\section{\emph{Cache to Json}}

\lstset{language=Python}
\begin{lstlisting}
import json
from collections import OrderedDict
from google.protobuf import text_format
import caffe.proto.caffe_pb2 as caffe_pb2      # 载入caffe.proto编译生成的caffe_pb2文件


caffeprototxt_path = "./deploy.prototxt"
calibletable_json_path = "./resnet18-cifar10-int8.json"
# load deploy.prototxt
net = caffe_pb2.NetParameter()
text_format.Merge(open(caffeprototxt_path).read(), net)
# load jsonfile
with open(calibletable_json_path, "r") as f:
    calible = json.load(f, object_pairs_hook=OrderedDict)
_scales = []
_mins = []
_maxs = []
_offsets = []
_new = OrderedDict()
items = calible.items()
for key, value in items:
    _scales.append(value['scale'])
    _mins.append(value['min'])
    _maxs.append(value['max'])
    _offsets.append(value['offset'])
for idx, _layer in enumerate(net.layer):
    _tempDict =  OrderedDict({
        "scale": _scales[idx],
        "min": _mins[idx],
        "max": _maxs[idx],
        "offset": _offsets[idx],
    }) 
    _new[_layer.name] =_tempDict
with open('resnet18-cifar10-int8-fixed.json', 'w') as f:
    json.dump(_new, f)
\end{lstlisting}

\section{\emph{system-user.dtsi}}

\begin{lstlisting}
/include/ "system-conf.dtsi"
/ {
    reserved-memory {
        #address-cells = <1>;
        #size-cells = <1>;
        ranges;
    
        nvdla_reserved: buffer@0x30000000 {
            compatible = "shared-dma-pool";
            no-map;
            reg = <0x30000000 0x10000000>;
        };
    };
};

&NV_nvdla_wrapper_0{
    compatible = "nvidia,nv_small";
    memory-region = <&nvdla_reserved>;
};
\end{lstlisting}

\section{\emph{Makefile}}

\begin{lstlisting}
obj-m := opendla.o

###append all of sources###
opendla-objs := nvdla_core_callbacks.o nvdla_gem.o scheduler.o engine.o bdma.o conv.o sdp.o cdp.o pdp.o rubik.o cache.o common.o engine_data.o engine_isr.o engine_debug.o
###########################


SRC := $(shell pwd)

all:
    $(MAKE) -C $(KERNEL_SRC) M=$(SRC)

modules_install:
    $(MAKE) -C $(KERNEL_SRC) M=$(SRC) modules_install

clean:
    rm -f *.o *~ core .depend .*.cmd *.ko *.mod.c
    rm -f Module.markers Module.symvers modules.order
    rm -rf .tmp_versions Modules.symvers
\end{lstlisting}

\section{\emph{opendla.bb}}

\begin{lstlisting}
SUMMARY = "Recipe for  build an external opendla Linux kernel module"
SECTION = "PETALINUX/modules"
LICENSE = "GPLv2"
LIC_FILES_CHKSUM = "file://COPYING;md5=12f884d2ae1ff87c09e5b7ccc2c4ca7e"

inherit module

SRC_URI = "file://Makefile \
            file://nvdla_core_callbacks.c \
            file://nvdla_gem.c \
            file://scheduler.c \
            file://engine.c \
            file://bdma.c \
            file://conv.c \
            file://sdp.c \
            file://cdp.c \
            file://pdp.c \
            file://rubik.c \
            file://cache.c \
            file://common.c \
            file://engine_data.c \
            file://engine_isr.c \
            file://engine_debug.c \
            file://common.h \
            file://dla_debug.h \
            file://dla_fw_version.h \
            file://dla_engine.h \
            file://dla_engine_internal.h \
            file://dla_err.h \
            file://dla_interface.h \
            file://dla_sched.h \
            file://engine_debug.h \
            file://nvdla_interface.h \
            file://nvdla_linux.h \
            file://opendla.h \
            file://opendla_initial.h \
            file://opendla_small.h \
            file://nvdla_ioctl.h \
            file://COPYING \
        "
S = "${WORKDIR}"

# The inherit of module.bbclass will automatically name module packages with
# "kernel-module-" prefix as required by the oe-core build environment.
\end{lstlisting}
