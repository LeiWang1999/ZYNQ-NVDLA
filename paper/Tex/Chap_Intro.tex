\chapter{绪论}\label{chap:introduction}

\section{研究背景及国内外研究现状}

卷积神经网络(CNN)现在是许多学科研究的热点之一,被广泛用于多种领域,特别是在模式识别、图像处理、计算机视觉等方面。但是卷积神经网络的主要问题在于计算量太大,特别是其中的卷积层,以Alex-Net为例,占用了90\%\citep{chen2016eyeriss} 以上的计算量,卷积神经网络的硬件加速逐渐成为一个热门的研究问题。由于卷积神经网络自身特点,层与层之间可以看做顺序执行,而层内则有着较高的并行性,因此提高层内计算的并行度成为加速卷积神经网络的一个重要方向。

在2009,Farabet等人提出来一种基于FPGA的CNN,该结构使用卷积单元来处理数据,并使用一个通用CPU来控制卷积单元\citep{farabet2009cnp}。但是由于FPGA资源的限制,该平台只实现了一个卷积核。如果计算需要多个卷积核,那么只能串行执行。2013年,Peemen等人实现了一个以存储为中心的CNN协处理器,它利用CNN大量内存访问的特点,在存储部分使用SRAM,而PE部分使用SIMD指令\citep{peemen2013memory}。2015年,清华大学的方睿等人,提出一种多级流水线的管道加速器方案。CPU通过PCIE通道提供数据并控制整个逻辑单元。近年来,中科院的陈天石等人提出来DianNao系列的加速器,目前是卷积神经网络硬件加速领域的较优的一种方案,可以实现多种结构的卷积网络,如MLP、CNN、DNN\citep{chen2014diannao}。纯硬件实现的卷积神经网络加速器通用性不好。尽管可以通过配置来实现更多的结构,但是它的灵活性远不如通用CPU。因此可重构加速器与通用CPU相结合的模式是一种高效地解决卷积神经网络加速问题的方案。

但是在这种定制结构中,CPU的选择具有极大的挑战。通常商业授权的IP会限制对指令集的修改,影响卷积神经网络加速器的实现效果。并且商业授权的IP通常需要高昂的授权费,不利于高校和个人的研究。因此需要一个开源的指令集架构来进行定制加速器的研究。

RISC-V是一种新的开源指令集架构(ISA)\citep{waterman2011risc},目前已经有了一个完整的硬件和硬件生态\citep{asanovic2014instruction},包括完整的指令集、相应的编译器、模拟器和工具链。利用开源的RISC-V处理器,研究人员可以方便地将可重构加速器整合进SOC,并拓展相应的指令集来实现加速器的软件接口。

% \section{设计目标及平台的选择}

% 本设计主要包含三个部分,RISC-V Core和卷积神经网络加速器,以及将两部分结合起来实现完整功能的SOC部分。卷积神经网络的硬件加速通常可以采用ASIC或者FPGA来实现,两者均是采用定制的硬件电路来加速算法。通常ASIC的性能跟高,功耗更低,但是由于成本过高,因此在本设计中采用FPGA作为硬件平台。

% 本设计在第五章中提出来一种使用 ZYNQ 芯片上运行 Linux 系统与 RISC-V 处理器共享内存的方法来解决一些 RISC-V 处理器缺少调试接口,不方便调试和下载程序的方法。考虑各方面因素,最终选取了 PYNQ-Z2 作为硬件平台。 PYNQ-Z2 上使用Xilinx ZYNQ 7020芯片。芯片包含了 13,300 个可编程逻辑单元,每个逻辑单元有4个6输入查找表和8个触发器。芯片有 630 KB 的块存储器,220 个 DSP 运算单元,逻辑资源相对丰富。开发板引出数十个 IO 口和一个 HDMI 接口,拥有 512MB 的 DDR3,足够该系统使用。

\section{研究意义及前景}

卷积神经网络作为目前深度学习的一个重要手段,由于计算量巨大的问题限制了其应用场景,通常需要在高性能的运算平台上进行模型的推演,因此主要被应用在主机场景,而非便携式应用。而利用卷积神经网络层内并行的特点,对卷积神经网络进行硬件加速,从而极大地提高了卷积网络的运行速度,使得便携式低功耗设备也可以进行卷积神经网络运算,极大地拓展了深度学习的应用场景。

\section{研究内容及结构安排}

本文主要在 FPGA 平台上设计并实现了一种卷积神经网络加速器结构,并通过 RISC-V 处理器完成流程控制和一部分运算任务。通过软件配置的方法来调整卷积神经网络加速器网络的结构,使其具有一定的通用性。通过对卷积神经网络模型的分析,划分计算任务。将不同的任务使用 FPGA 进行加速,分析加速的效果,最终得到一种效果最好的加速器优化方案。

本文共有七个章节,论文结构安排如下:

第一章:主要介绍了课题的研究背景及研究意义,结合国内外研究现状,分析使用 FPGA 加速神经网络算法的所遇到的问题,并提出本文的研究方向。

第二章:主要对本文研究所涉及到的相关技术背景的介绍,首先介绍了人工神经网络,并对本文所使用的算法——YOLO V2 算法进行了详细的分析,同时介绍了 RISC-V 处理器相关的背景以及在本系统中为何选用 RISC-V 处理器。

第三章:总体介绍了系统的设计思路:先对 YOLO V2 算法进行了详细的分析,结合 FPGA 与 CPU 各自的特点,对整个神经网络算法的加速进行了任务划分,并给出本设计所使用的逻辑结构。

第四章:对 CNN 加速器进行详细设计及优化,介绍了使用 HLS 设计 CNN 加速器过程中的优化方案以及具体应用情况。

第五章:由于 CNN 加速器需要配合 RISC-V 软核进行异构运算,因此本章将 CNN 加速器作为 RISC-V 处理器的外设,设计了一个通用 RISC-V SOC,介绍了该 SOC 的硬件架构以及详细设计过程。

第六章:主要计算分析了 CNN 加速器的资源消耗,结合常见的运算平台,设计了3组典型的对照环境,将本设计中的异构计算系统与其分别进行对比,分析了该异构计算系统的优缺点。

第七章:总结全文的研究内容,对本课题中可以进一步研究的方向进行讨论。



