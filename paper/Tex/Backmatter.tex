%---------------------------------------------------------------------------%
%->> Backmatter
%---------------------------------------------------------------------------%
% \chapter{作者简历及攻读学位期间发表的学术论文与研究成果}

% \textbf{本科生无需此部分}。

% \section*{作者简历}

% \subsection*{casthesis作者}

% 吴凌云,福建省屏南县人,中国科学院数学与系统科学研究院博士研究生。

% \subsection*{ucasthesis作者}

% 莫晃锐,湖南省湘潭县人,中国科学院力学研究所硕士研究生。

% \section*{已发表(或正式接受)的学术论文:}

% {
% \setlist[enumerate]{}% restore default behavior
% \begin{enumerate}[nosep]
%     \item ucasthesis: A LaTeX Thesis Template for the University of Chinese Academy of Sciences, 2014.
% \end{enumerate}
% }

% \section*{申请或已获得的专利:}

% (无专利时此项不必列出)

% \section*{参加的研究项目及获奖情况:}

% 可以随意添加新的条目或是结构。

\chapter[致谢]{致\quad 谢}\chaptermark{致\quad 谢}% syntax: \chapter[目录]{标题}\chaptermark{页眉}
\thispagestyle{noheaderstyle}% 如果需要移除当前页的页眉
%\pagestyle{noheaderstyle}% 如果需要移除整章的页眉

时光易逝,随着毕业论文的定稿四年的本科求知时光也就要结束了,行文至此,心中满是感恩。

细细想来,这一路的打怪升级要始于三年前的夏天。我有幸加入了学院的硬件部,跟随在已经推免至东南大学的周玉乾学长后面学习,在 FPGA 学习上,学长给了我莫大的帮助;当时一同在硬件部学习的王镇、吴佳昱两位同学也成了我参加很多比赛的队友,一起在创新工作室学习的葛志来同学也带给我一些欢乐,他们分别推免至清华大学、南京大学、复旦大学,对他们都有了好的去处我作为朋友很是开心;在创新工作室的前辈们,尤其是陈宇明学长,跟随在陈宇明学长身后运维南工在线的时间使我掌握了很多前沿的技术,他们在我的毕业设计中被灵活运用。

说起本次毕业设计,已经退休的包亚萍教授利用自己的科研经费为本毕业设计采购了开发板卡,包老师严谨的治学态度、精益求精的工作作风深刻地影响着我的本科生活;朱艾春老师是本次毕业设计的指导老师,其认真负责,指导了很多本论文的写作;中国科学院信息工程研究所的王兴宾博士曾经参与过 NVDLA 流片工作,与他交流解决了我的一些困惑;在计算所的许浩博师兄、肖航师兄、徐宇师兄、卢美璇师姐以及韩银和老师,在本次毕业设计他们提供了包括但不限于研究方向、采购开发板卡、GPU 服务器、论文写作等支持与指导。

本科不是终点而是起点,山鸟与鱼不同路。回忆起四年前一只脚迈进大学之门的时候,感到的更多的是不确定和迷茫,而现在即将迈出下一步的时候,未来依然是如此的不确定,但是期待却又更多了一些。感谢所有给予过我帮助的老师和同学,使我拥有了无比丰富精彩的大学时光!

\cleardoublepage[plain]% 让文档总是结束于偶数页,可根据需要设定页眉页脚样式,如 [noheaderstyle]
%---------------------------------------------------------------------------%
